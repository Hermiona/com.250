\documentclass[12pt,a4paper,oneside]{article}

\usepackage[margin=3cm]{geometry}

\usepackage{hyperref}
\hypersetup{
    pdftitle={COM 250, Mobile Application Development},%
    pdfauthor={Toksaitov Dmitrii Alexandrovich},%
    pdfsubject={Syllabus},%
    pdfkeywords={COM;}{250;}{syllabus;}{mobile;}{application;}{development},%
    colorlinks,%
    linkcolor=black,%
    citecolor=black,%
    filecolor=black,%
    urlcolor=black
}

\newcommand{\R}[1]{\uppercase\expandafter{\romannumeral #1\relax}}

\begin{document}

    \title{COM 250, Mobile Application Development}
    \author{
        American University of Central Asia\\
        Software Engineering Department
    }
    \date{}
    \maketitle

    \section{Course Information}

        \begin{description}
            \item[Course ID]\hfill\\
                COM 250, 3371
            \item[Course Repositories]\hfill\\
                \url{https://github.com/auca/com.250}
            \item[Place]\hfill\\
                AUCA, laboratory G30
            \item[Time]\hfill\\
                Monday 9:25\\
                Friday 9:25
        \end{description}

    \section{Prerequisites}

        COM 112, Programming \R{2}. Object Oriented Design and GUI Programming

        \section{Contact Information}

            \begin{description}
                \item[Instructor]\hfill\\
                    Toksaitov Dmitrii Alexandrovich\\
                    \href{mailto:toksaitov_d@auca.kg}{toksaitov\_d@auca.kg}
                \item[Office]\hfill\\
                    AUCA, room 315
                \item[Office Hours]\hfill\\
                    Monday 12:00--14:00\\
                    Wednesday 14:00--15:00\\
                    Friday 12:00--14:00
            \end{description}

    \section{Course Overview}

        This course introduces students to development tools and APIs to build
        applications for the Google Android operating system. Students will
        learn how to build unique interactive user interfaces for multi-touch
        mobile devices. The course covers object-oriented design using the
        Model-View-Controller paradigm, the Java programming language for the
        Dalvik virtual machine, the Android development framework, device
        emulators, and application build tools. Other topics include
        multi-threading, power and performance considerations, the accelerated
        2-D and 3-D graphics APIs.

    \section{Topics Covered}

        \begin{itemize}
            \item Development tools (Android Studio, SDK, device emulators)
            \item A Java language crash course
            \item App. fundamentals (activities, services, content providers)
            \item User interface elements
            \item Graphics and animation
            \item Data storage
            \item Connectivity
            \item Media and camera
            \item Working with device sensors
            \item Publishing and distributing applications
        \end{itemize}

    \section{Practice Tasks}

        Students are required to finish 10 practice tasks. The tasks are based
        on topics discussed during lectures. Each task should be finished during
        the class to receive a grade.

    \section{Course Projects}

        Each student has to select a topic of interest and develop an app for
        the Android platform. The challenge of the project is to maintain a
        certain level of quality for the application to be able to publish it to
        end users on Google Play Store at the end of the course.

    \section{Final Exam}

        At the end of the course, students have to take a final examination in a
        form of a quiz with a number of multiple choice questions on topics
        discussed during classes.

    \section{Reading}

        \begin{enumerate}
            \item Introduction to Android Application Development: Android
            Essentials, 5th Edition by Joseph Annuzzi Jr., Lauren Darcey, Shane
            Conder (ISBN: 978-0134389455)
            \item Java: A Beginner's Guide, 6th Edition by Herbert Schildt (ISBN:
            978-0071809252)
        \end{enumerate}

            \subsection{Supplemental Reading}

                \begin{enumerate}
                    \item Design Patterns: Elements of Reusable Object-Oriented
                    Software by Erich Gamma, Richard Helm, Ralph Johnson, John
                    Vlissides (AUCA Library Call Number: QA 76.64 D47 1995,
                    ISBN: 978-0201633610)
                    \item Refactoring: Improving the Design of Existing Code by
                    Martin Fowler, Kent Beck, John Brant, William Opdyke, Don
                    Roberts (AUCA Library Call Number: QA76.76.R42 F695 1999,
                    ISBN: 978-0201485677)
                \end{enumerate}

    \section{Grading}

        \begin{itemize}
            \item Practice tasks (30\%)
            \item Course project (40\%)
            \item Final examination (30\%)
        \end{itemize}

        \begin{itemize} \itemsep-10pt \parskip0pt \parsep0pt
            \item[--] 90\%--100\%: A\\
            \item[--] 80\%--89\%: A-\\
            \item[--] 70\%--79\%: B+\\
            \item[--] 65\%--69\%: B\\
            \item[--] 60\%--64\%: B-\\
            \item[--] 56\%--59\%: C+\\
            \item[--] 53\%--55\%: C\\
            \item[--] 50\%--52\%: C-\\
            \item[--] 46\%--49\%: D+\\
            \item[--] 43\%--45\%: D\\
            \item[--] 40\%--42\%: D-\\
            \item[--] Less than 39\%: F
        \end{itemize}

    \section{Rules}

        Students are required to follow the rules of conduct of the Software
        Engineering Department and American University of Central Asia.

        Team work is NOT encouraged. Equal blocks of code or similar structural
        pieces in separate works will be considered as academic dishonesty and
        all parties will get zero for the task.

\end{document}
